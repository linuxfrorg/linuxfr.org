\documentclass[a4paper]{recipe}

\usepackage[T1]{fontenc}
\usepackage[utf8]{inputenc}
\usepackage[frenchb]{babel}

\newcommand{\bsi}[2]{%
  \fontencoding{T1}\fontfamily{pbs}\fontseries{xl}\fontshape{n}%
  \fontsize{#1}{#2}\selectfont}

\renewcommand{\inghead}{\textbf{Ingrédients}:\ }
\renewcommand{\rechead}{\centering\bsi{24pt}{30pt}}

\makeatletter
\renewcommand*\l@subsubsection{\@dottedtocline{3}{3em}{0em}}
\makeatother

\thispagestyle{empty} 

\begin{document}
Recette libre d'Étienne Loks distribuée sous licence WTFPL
\recipe{Cake automnal}
\ingred{250 grammes de farine;
150 grammes de sucre non raffiné;
150 grammes de beurre salé\footnote{car vous n'avez pas de beurre doux
chez vous};
3 \oe ufs;
1 sachet de levure chimique;
50 grammes de poudre d'amande;
3-4 figues moelleuses;
une petite poignée de cerneaux de noix;
une petite poignée de graines de tournesol;
des pistaches non salées\footnote{si comme moi vous n'avez pas oublié
de les acheter};
une demi-poire;
}

Faire fondre le beurre. Travailler au fouet le beurre fondu et le sucre
jusqu'à ce que le mélange blanchisse\footnote{comme dans toute recette
de cuisine comprendre : jusqu'à ce que vous en ayez marre car cela ne
blanchira pas}. Incorporer les \oe ufs un à un puis la farine, la
levure et la poudre d'amande. Une fois le mélange homogène ajoutez les
figues découpées en morceaux, les cerneaux de noix un peu cassés\footnote{
résistez à l'envie d'ajouter du chèvre}, la poire découpée\footnote{
résistez à l'envie d'ajouter du bleu - oui je suis plus
cake salé que cake sucré à la base}, les graines de tournesols et les
pistaches. Mettez tout fruit sec (ou pas) hors de votre portée, le
cake est déjà trop riche. Beurrez le moule à cake, versez la pâte,
cuire environ 45 minutes (en vérifiant la cuisson avec le couteau).


\end{document}

